\section*{はじめに}

{\center\large\bf -- Under Construction --\\}

この授業は前学期の「情報基礎」に続く授業です.
Pythonの文法は一通り覚えたって受講生を対象に,
データサイエンス・人工知能を視野に入れつつ,
もうちょっとプログラミングを進める.

プログラミングは人間活動一般に通じる教義を含んでいると思う.

\begin{itemize}
\item小さなものを組み合わせて大きなXXXを作るには?
\item複雑であっても安全・安心なXXXを作るには?
\item動作が正確で素早く軽やかなXXXを作るには?
\itemみんなの役に立つXXXを作るには?
\end{itemize}
%
XXXにはもちろん「コンピュータプログラム」が当てはまることを想定してるが,
他に日常生活の些細な事柄から大企業の国際的なプロジェクトまでいろんなものがXXXに当てはまって
関係者たちの一番の目標になっているものもあると思う.
%分野に関わる人たちは分野ごと,日夜研鑽を重ねているに違いない.
%プログラミングを学ぶことは他の人間活動に敷衍する.

プログラミングが苦手な人たちにとってプログラムはわけのわからない呪文であるに違いない.
1文字違ったくらいでエラーになったり,ひどく間違った答えを出したりして,がっかりする.
% 1文字まちがいのシンタックスエラーはエラーとしたら最も修正しやすいエラー.
でも,コンピュータサイエンスの先駆者たちのこんな言葉を聞くと,はっとさせられるんじゃないかな.

\begin{quote}
Programs must be written for people to read,
and only incidentally for machines to execute.
「計算機プログラムの構造と解釈」の序文から.
\end{quote}

他人が読んで理解できるようなプログラムを書いているか?

動けばいいってうそぶく人もいるが,そういうつもりで作ったプログラム,あるいは,
プログラム以外のプロダクトは本当に満足できるレベルで「動く」って言えるのか.自問してみよう.
プログラムが読めない,書けない,わからない時,
自分がコンピュータに実行させようとしている作業の内容を
具体的に理解できてなく,そのため説明できず,定式化できないってことが多いはず.
プロダクトも同じ.

%一般の仕事との関連性
最初は動かない,バグのあるプログラムであっても,
それをきちんと動ようにしようとする過程で,
曖昧だった概念が洗練されたり,
最初のアイデアにあった誤りに気がついたり,
さらにいいアプローチが見つかったりする.
これって他の仕事も同じじゃね?
%他の仕事にしても同じこと.
プログラミングの方法論は一般的な人間活動の方法論に敷衍する.%って言ったら大袈裟か?

%第一勘の一筆書きでプログラムは出来上がりなんてことはほとんどない.
%完成したと思ったプログラムでも,いろいろ,直したいところ,改良したいところ,
%付け加えたいところが出てくる.

プログラミングを学ぶには言語を設定する必要があり,この授業ではPythonを選ぶわけだが,
たくさんあるプログラミング言語のうち,Pythonは
初心者が学ぶ言語としてはベストのものではないと思う.
Pythonは2023年現在,バージョン3が主流だが,その前の世代のバージョン2とは互換性がない.
人気を博しつつ改良が加えられて今の姿になったというのが偽らざるストーリー.
後から付け加えられた一貫性を欠く部分や機能があるものの,
反省なく最初の姿のままって頑固者のとは違って
趨勢に合わせて柔軟に姿形を変えてきた優等生ということもできるかもしれない.
ベストじゃなくてもPythonは悪い選択ではない.

授業は「Python ``を''学ぶ」よりも,
「Python ``で''学ぶ」にしたいです.

%生まれながらにプログラマーなんて人はいない.
%トレーニングなしにいいプログラーにはなれない.
%トレーニングしましょう.

学習を進めるために,オンラインのサイトを設けています.

\begin{itemize}
\item https://py99.melt.kyutech.ac.jp
\item https://qa.melt.kyutech.ac.jp
\end{itemize}

いずれのサイトもアカウント・パスワードが必要です.授業中に説明します.
